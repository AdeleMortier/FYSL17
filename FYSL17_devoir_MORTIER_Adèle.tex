\documentclass[a4paper,10pt]{article}
\usepackage[utf8]{inputenc}
\usepackage[french]{babel}
\usepackage{frbib}
\usepackage{french}
\usepackage{graphicx}
\usepackage{geometry}
\geometry{
	a4paper,
	left=20mm,
	top=20mm,
}


\title{ FYSL17 -- Approches sociocritiques de la littérature contemporaine\\ \vspace{0.3cm}
	 \small La question de la mésalliance parentale dans \textit{La Route des Flandres}}
\author{Adèle Mortier}

\begin{document}
\maketitle
\nocite{*}
\section*{Introduction}
\section{Moteurs de la mésalliance}
\section{Oppositions entre époux}
	\subsection{Le schéma linéaire : les ancêtres Reixach et les parents de Georges}
	\subsection{Le schéma triangulaire : le couple officiel et le couple adultère}
\section{Nuances au sein des classes}
	\subsection{Les "aristocrates" : de Reixach et Sabine}
	\subsection{Les "roturiers" : le père de Georges, Corinne et Iglésia}
\section{Produits de la mésalliance et phénomènes de transferts sociaux}
	\subsection{Position problématique de Georges}
	\subsection{Fantasmes et "polymorphisme social"}

\section*{Conclusion}

\begin{center}
	\footnotesize
	\begin{minipage}{0.7\textwidth}
		citation
	\end{minipage}
\end{center}
\medskip

\bibliographystyle{frcomplet}
\bibliography{bibliography}

\end{document}
